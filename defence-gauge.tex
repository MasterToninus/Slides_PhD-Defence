%+----------------------------------------------------------------------------+
%| SLIDES: 
%| Chapter: Results of the paper with M. Zambon
%| Author: Antonio miti
%| Event: PHD preliminary Defence
%+----------------------------------------------------------------------------+

%- HandOut Flag -----------------------------------------------------------------------------------------
\newif\ifHandout

%- D0cum3nt ----------------------------------------------------------------------------------------------
\documentclass[beamer,10pt]{standalone}   
%\documentclass[beamer,10pt,handout]{standalone}  \Handouttrue  

%- HandOut Flag -----------------------------------------------------------------------------------------
\ifHandout
	\setbeameroption{show notes} %print notes   
\fi

	
%- Packages ----------------------------------------------------------------------------------------------
\usepackage{custom-style}

%--Beamer Style-----------------------------------------------------------------------------------------------
\usetheme{toninus}






%---------------------------------------------------------------------------------------------------------------------------------------------------
%- D0cum3nt ----------------------------------------------------------------------------------------------------------------------------------
\begin{document}
%------------------------------------------------------------------------------------------------


%TITOLO DA AGGIORNARE
%-------------------------------------------------------------------------------------------------------------------------------------------------
%\subsection{Commutative diagram with gauge transformations and HCMM}
%-------------------------------------------------------------------------------------------------------------------------------------------------


%-------------------------------------------------------------------------------------------------------------------------------------------------
\begin{frame}[fragile]{Compatibility between gauge transformations and comoment maps}
	%
	Consider $(M,\omega)$ \alert{symplectic mfd.}
	\only<5-11>{ and \alert{\underline{prequantizable}} ($S^1$-bundle $P$, connection $\theta$)}
	%
	\begin{center}
			\includestandalone[width=.8\textwidth]{Pictures/Frame_BigDiagram_symplectic}
	\end{center}
	%
	\vspace{-2em}
	\begin{minipage}[t][1.7cm][t]{\textwidth}
	\begin{itemize}
		\only<1-4>{
			\item<2-> Given a Symp. mfd. $(M,\omega)$ there is a naturally associated Poisson algebra ...
			\item<3-> .\alert<+>{... and also a Lie Algebroid}.
			\item<4-> A Lie algebroid is a "controlled" $\infty$-dimensional Lie algebra;
		}
		\only<5-6>{
			\item<5-> Prequantization Bundle $S^1\hookrightarrow P \to M$ with connection $\theta$,
			\item<5-> "infinitesimal quantomorphisms" $Q(P,\theta):=\lbrace Y \in \mathfrak{X}(P)~|~ \mathcal{L}_Y \theta =0 \}$.
		}
		\only<7-11>{
		\item<7-> Consider a deformed structure $\tilde{\omega}= \omega + d B$ with $B\in C^\infty(M)$;
		\item<9-> There is a natural isomorphism in the Lie Alg.oids category,
		\item<11-> Considering $\mathfrak{g}\circlearrowleft M$ preserving $\omega$ and $\tilde{\omega}$ ...
		}
		\item<12-> Neglect the prequantization...
		\vspace{-1em} 
			\begin{displaymath}
				\begin{tikzcd}
					\Psi~:&[-1em] C^{\infty}(M)_\omega \ar[r,"\Psi"]& \Gamma(TM\oplus \mathbb{R})_\omega
					\\[-2em]
					& f \ar[r,mapsto] & \binom{\mathscr{v}_f}{f}
				\end{tikzcd}
			\end{displaymath}
	\end{itemize}
	\end{minipage}
	\vfill
	\tcbset{colback=white,
	colbacktitle=white,
	colframe=red!70!black,
	boxrule=1pt,
	colupper=red!70!black,
	arc=15pt,
	}
	\begin{minipage}[t][1.7cm][t]{\textwidth}
	\only<6>{ 
		\begin{tcolorbox}[enhanced,frame hidden,borderline={0.5pt}{0pt}{blue}]
			\color{blue}{
			The left square and right triangles commutes!
			}
		\end{tcolorbox}
	}
	\only<11>{
		\begin{tcolorbox}[enhanced,frame hidden,borderline={0.5pt}{0pt}{blue}]
			\color{blue}{
			The left square and right triangle commute!
			}
		\end{tcolorbox}
	}
	\only<12>{
		\vspace{-.75em}
		\begin{tcolorbox}[enhanced,frame hidden,borderline={0.5pt}{0pt}{blue}]
			\color{blue}{
			The central pentagon commutes!
			}
		\end{tcolorbox}
	\vfill
		\vspace{-.75em}
		\begin{center}
		\tcbox[enhanced,frame hidden,borderline={0.5pt}{0pt}{red,dashed}]{	
			\alert{
			\faQuestionCircle \qquad
				{What happens in the higher (n-plectic) case?}
			\qquad \faQuestionCircle		
			}
		}
		\end{center}
	}
	\end{minipage}
	%
\end{frame}
\note[itemize]{
	\item The horizontal embedding is  $f \mapsto (v_f,f)$;
	\item Vertical maps are also known as \emph{Gauge transformations}
	\item upshot: 
	\begin{enumerate}
		\item 
	\end{enumerate}
}
%-------------------------------------------------------------------------------------------------------------------------------------------------


%-------------------------------------------------------------------------------------------------------------------------------------------------
\begin{frame}[fragile]{Embedding observables $L_\infty$-algebra into Vinogradov $L_\infty$-algebra}
	Consider now $\omega$ \alert{$n$-plectic}
	\vfill
	\begin{center}
		\includestandalone[width=.8\textwidth]{Pictures/Frame_Embedding_Diagram_k-plectic_V1}
	\end{center}
	\vfill
	\only<1-3>{
	\begin{itemize}
		\item<2-> Higher analogue of the Courant algebroid $\rightsquigarrow$ \alert{\emph{Vinogradov algebroid}}
			\begin{displaymath}
			E^n = \left(TM\oplus\bigwedge^{k-1}T^\ast M \right)
			\end{displaymath}
		\item<3->  Vin. alg.oids are $NQ$-manifolds ($L_\infty$-algebroids).
		\\ Associated $L_\infty$-algebra on the graded vector space
		\begin{displaymath}\label{eq:VSpace}
			{\mathcal{V}^k} =
			\begin{cases}
				\mathfrak{X}(M)\oplus \Omega^{n-1}(M)  &\quad k=0,\\
				\Omega^{n-1+k}(M) &\quad -n+1 \leq k < 0.
			\end{cases}
		\end{displaymath}
	\end{itemize}
	}


	\only<4->{
	\begin{thmblock}[Embedding of $L_\infty$-algebras  $\Psi:L_\infty(M,\omega)\hookrightarrow L_{\infty}(E^n,\omega)$\quad \cite{Miti2021}.]
	\begin{itemize}[leftmargin=0pt]
		\item[$\cdot$]<4-> 
			consider the graded vector subspace $\mathcal{A}$
			\begin{displaymath}
			\mathclap{
			{\mathcal{A}^k} =
			\begin{cases}
		\left.\left\lbrace
		\binom{X}{\alpha}\in \mathfrak{X}(M)\oplus \Omega^{n-1}(M)
		~ \right\vert ~
		\iota_X \omega = -d \alpha\right\rbrace
&\quad k=0,\\
				\Omega^{n-1+k}(M) &\quad -n+1 \leq k < 0.
			\end{cases}			
			}
			\end{displaymath}						
		\item[$\cdot$]<5-> 
			restrict the two $L_\infty$-structures to $\pi$ and $\mu$ on $\mathcal{A}$
		\item[$\cdot$]<6->

			$L_\infty(M,\omega) \cong 
				(\mathcal{A},\pi) \color{blue}\cong\color{black}
				(\mathcal{A},\mu) \hookrightarrow
				L_\infty(E^n,\omega)$
	\end{itemize}
	\end{thmblock}
	}
	\only<6->{
		\tikz[overlay,remember picture]
		{
			\node[rounded corners,
                 fill=orange!1,draw=orange!30,anchor=base]            
            	 (base) at ($(current page.east)-(2.25,4)$) [rotate=-0,text width=3cm,align=center] { \footnotesize{\color{red}{
            	 Complete proof\\
            	 \faWarning ~ up to $n\geq 4$! ~ \faWarning 
            	 }}};
		}			
	}

		

\end{frame}
\note{}
%-------------------------------------------------------------------------------------------------------------------------------------------------


%-------------------------------------------------------------------------------------------------------------------------------------------------
\begin{frame}{Compatibility with Gauge transformations}
	Consider now $\omega$ \alert{n-plectic} \quad and \alert{$\tilde{\omega}=\omega + d B$}:
	\vfill
	%
	\begin{center}
		\includestandalone[width=.8\textwidth]{Pictures/Frame_Gauge_Diagram_k-plectic}
	\end{center}	
	%
	\vfill
	%
	

	\begin{itemize}
	\only<1-3>{
		\item<2-> Vinogradov alg.oids w.r.t cohomologous twisting closed forms are isomorphic.
		\item<3-> Induced isomorphism at the level of $L_\infty$-algebras
	}
	\only<4->{
		\item<4-> Consider a Lie algebra action $\mathfrak{g}\to \mathfrak{X}(M)$ admitting HCMM w.r.t $\omega$ and $\tilde{\omega}$
	}
	\end{itemize}

	\vfill
	\tcbset{colback=white,
		colbacktitle=white,
		colframe=blue!70!black,
		boxrule=1pt,
		colupper=blue!70!black,
		arc=15pt,
		}
	\onslide<5->{
	\begin{tcolorbox}[sidebyside,righthand width=.75\linewidth]
		Thm: \cite{Miti2021}
		\tcblower
		\color{blue}
The central square commutes. 
			\\\emph{(On the nose, not "up to homotopies")}.
	\end{tcolorbox}	
		\tikz[overlay,remember picture]
		{
			\node[rounded corners,
                 fill=orange!1,draw=orange!30,anchor=base]            
            	 (base) at ($(current page.east)-(1.75,4)$) [rotate=-0,text width=3cm,align=center] { \footnotesize{\color{red}{
            	 Complete proof\\
            	 \faWarning ~ up to $n\geq 4$! ~ \faWarning 
            	 }}};
		}		
	
	
	}
	

\end{frame}
\note[itemize]{
	\item Our results can be seen as a tiny step toward  undestanding the analogue of prequantization in the setting of multisymplectic geometry (hence field theory).
}
%-------------------------------------------------------------------------------------------------------------------------------------------------




%----------------------------------------------------------------------------------------------------------------------------------
\end{document}
%----------------------------------------------------------------------------------------------------------------------------------




