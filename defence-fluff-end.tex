%+----------------------------------------------------------------------------+
%| SLIDES: 
%| Chapter: Brief introduction to multisymplectic geometry and Homotopy momap
%| Author: Antonio miti
%| Event: PHD preliminary Defence
%+----------------------------------------------------------------------------+

%- HandOut Flag -----------------------------------------------------------------------------------------
\newif\ifHandout

%- D0cum3nt ----------------------------------------------------------------------------------------------
\documentclass[beamer,10pt]{standalone}   
%\documentclass[beamer,10pt,handout]{standalone}  \Handouttrue  

%- HandOut Flag -----------------------------------------------------------------------------------------
\ifHandout
	\setbeameroption{show notes} %print notes   
\fi

	
%- Packages ----------------------------------------------------------------------------------------------
\usepackage{custom-style}




%--Beamer Style-----------------------------------------------------------------------------------------------
\usetheme{toninus}
\usetikzlibrary{positioning}




%---------------------------------------------------------------------------------------------------------------------------------------------------
%- D0cum3nt ----------------------------------------------------------------------------------------------------------------------------------
\begin{document}
%------------------------------------------------------------------------------------------------


%-------------------------------------------------------------------------------------------------------------------------------------------------
\subsection{Going higher}
%-------------------------------------------------------------------------------------------------------------------------------------------------

%-------------------------------------------------------------------------------------------------------------------------------------------------
\begin{frame}[t]{Geometric Mechanics: Take Away Message}
	\begin{itemize}
		\item[•] \alert{States} are encoded by a \alert{symplectic manifold}.
		\begin{itemize}
			\item[-] Smooth manifolds arise naturally in the description of mechanical systems.
			\item[-] Constraints can be enforced intrinsically
			\item[-] Geo. Mech. yields an inherent intuition of diff. geom. structures.
		\end{itemize}
		\item<2->[•] \alert{Observables} quantities are encoded by \alert{smooth functions}.
		\begin{itemize}
			\item[-] The symplectic structure prescribe how to associate a vec. field to any observable.
			\item[-] Time evolution is obtained by integrating the flow along an Ham. v.f.
		\end{itemize}				
		\item<3->[•] \alert{Continuous symmetries} are encoded by \alert{symplectic Lie group actions}.
		\begin{itemize}
			\item[-] A relevant class of symmetries can be reconstructed from observables quantities via \emph{comomentum maps}.
			\item[-] Their importance lies in the existence of corresponding conserved quantities and the reduced description arising from them.
		\end{itemize}				
	\end{itemize}
	%Symplectic geometry is central
	\vfill
	%
	\onslide<4->{
		\begin{disclaimerbox}[we glossed on some technical details]
			\resizebox{.85\textwidth}{!}{
				\begin{columns}
					\begin{column}[T]{0.65\textwidth}
						\begin{itemize}
							\item[-] \color{black!80!white} Galilean / Special / General relativity (observer)
							\item[-] Holonomics / Non-holonomics constraints
							\item[-] Conservative / Dissipative system
							\item[-] Discrete / \alert{Continuous} degrees of freedom.
						\end{itemize}
					\end{column}					
					\begin{column}[T]{0.5\textwidth}
						\begin{itemize}
							\item[-] \color{black!80!white} Classical / Quantum observability
							\item[-] Lagrangian / Hamiltonian dynamics
							\item[-] Velocities $\neq$ Momenta
							\item[-] Local / Global flow integrability
						\end{itemize}
					\end{column}							
				\end{columns}
			}				
		\end{disclaimerbox}	
	}

\end{frame}
\note[itemize]{
	\item è centrale nel senso che è l'arena  in cui descrivere molti sistemi ideali
	\item richiede estensioni e generalizzazioni per codificare casi di studio piu' realistici e/o generali

	\item geometry is an intrisic part of mechanics. 
	The set of configurations has a natural geometric structure, constraints are automatically satisfied by this choice.
	\item {\bf descibing states as points of a manifold is the principal premise of geometric mechanics.}
	\item Conf. space and Phase space are manifolds (higher dimensional generalization of surfaces)
	\item A large class of constraints can be enforced intrinsically by the topology (shape) of the manifold.


	\item an important trait of geometric mechanics is its intuitive nature: structures and time evolution of mechanical systems can be illustrated by visualizing configuration and phase space.
	Practilly working in Geo. Mech. often means to exploit the inherent geometric intuition.
	
	\item an important characteristic is its emphasis on laying a rigourous mathematical foundation on which one can build upon mechanics. 
	While Newtonian mechanics is highly desciptive it does not reveal the underlying structure.
	In geo. mech. computations are structural arguments which provide insight into the frabic they represent. 

	\item Huge disclaimer, in the above presentation we glossed out on many details...

}

%-------------------------------------------------------------------------------------------------------------------------------------------------




%-------------------------------------------------------------------------------------------------------------------------------------------------
\begin{frame}[t,fragile]{Multisymplectic geometry in a nutshell}
	\begin{block}{Historical motivation}
		Mechanics: geometrical foundations of \textit{(first-order)} field theories.
	\end{block}
	\vfill	
	\begin{table}
		\only<2->{
		\begin{tabular}{|p{0.2\textwidth}|p{0.3\textwidth}|p{0.35\textwidth}|} 
            \hline
            \parbox[][20pt][c]{0.2\textwidth}{mechanics} & \multicolumn{2}{c|}{geometry} \\
            \hline
            \parbox[][20pt][c]{0.2\textwidth}{phase space} & symplectic manifold & \only<3->{multisymplectic manifold} \\[.25em]
            \parbox[][20pt][c]{0.2\textwidth}{classical \\ observables} & Poisson algebra & \only<3->{$L_\infty$-algebra} \\[.25em]
            \parbox[][20pt][c]{0.2\textwidth}{symmetries} &  group actions admitting comoment map & 
            \only<3->{group actions admitting 
				\tikz[baseline,remember picture]{\node[rounded corners,
	                        fill=orange!10,draw=orange!30,anchor=base]            
	            			(target) {homotopy comomentum map};
	            }
            }
            \\
            \hline
  \multicolumn{1}{c}{}
            &  
            \multicolumn{1}{@{}c@{}}{$\underbrace{\hspace*{.3\textwidth}}_{\text{point-like particles systems}}$} 
            &
            \multicolumn{1}{@{}c@{}}{\only<4->{$\underbrace{\hspace*{.3\textwidth}}_{\text{field-theoretic systems}}$}} 
		\end{tabular}
		}
	\end{table}		
	\vfill
	\onslide<4->{
		\begin{block}{Scope of the thesis}
			\begin{itemize}
				\item[$\bullet$] Develop theory of 
					\tikz[baseline,remember picture]{\node[rounded corners,
	                        fill=orange!10,draw=orange!30,anchor=base]            
	            			(base) {homotopy comomentum maps};
	            	}
	            \item[$\bullet$] produce new meaningful examples.
			\end{itemize}
		\end{block}
		%
        \begin{tikzpicture}[overlay,remember picture]
			\path[->]<4-> (base.east) edge[bend right](target.south east);
		\end{tikzpicture}
	}
	%
\end{frame}
\note[itemize]{
	\item Historically, the interest in multisymplectic manifolds, has been motivated by the need for understanding the geometrical foundations of first-order classical field theories.
	The key point is that, just as one can associate a symplectic manifold to an ordinary classical mechanical system (e.g. a single
point-like particle constrained to some manifold), it is possible to associate a multisymplectic
manifold to any classical field system (e.g. a continuous medium like a filament or a fluid). See frame Extra-\ref{Frame:Ms-Field-Mechanics} 
	
	\item General ideas basic parallelisms with caveats
	\item caveat: points in multiphase spaces are not states
	\item the table hides the duality between geometric and algebraic approaches to the problem.
	\item 
}
%-------------------------------------------------------------------------------------------------------------------------------------------------












%----------------------------------------------------------------------------------------------------------------------------------
\end{document}
%----------------------------------------------------------------------------------------------------------------------------------







