%+----------------------------------------------------------------------------+
%| SLIDES: 
%| Chapter: Results of the paper with L. Ryvkin
%| Author: Antonio miti
%| Event: PHD preliminary Defence
%+----------------------------------------------------------------------------+

%- HandOut Flag -----------------------------------------------------------------------------------------
\makeatletter
\@ifundefined{ifHandout}{%
  \expandafter\newif\csname ifHandout\endcsname
}{}
\makeatother

%- D0cum3nt ----------------------------------------------------------------------------------------------
\documentclass[beamer,10pt]{standalone}   
%\documentclass[beamer,10pt,handout]{standalone}  \Handouttrue  

\ifHandout
	\setbeameroption{show notes} %print notes   
\fi

	
%- Packages ----------------------------------------------------------------------------------------------
\usepackage{custom-style}

%--Beamer Style-----------------------------------------------------------------------------------------------
\usetheme{toninus}




\renewcommand{\action}{\circlearrowleft}

\newcommand*{\TakeFourierOrnament}[1]{{%
\fontencoding{U}\fontfamily{futs}\selectfont\char#1}}
\newcommand*{\danger}{\TakeFourierOrnament{66}}







%---------------------------------------------------------------------------------------------------------------------------------------------------
%- D0cum3nt ----------------------------------------------------------------------------------------------------------------------------------
\begin{document}
%------------------------------------------------------------------------------------------------

%-------------------------------------------------------------------------------------------------------------------------------------------------
\begin{frame}[fragile]{Problem: Geometric interpretation of cohomological obstructions}\label{frame:introcohoobstruction}
	Given a multisymplectic group action $\vartheta:G\action (M,\omega)$ 
	
	How to determine if $\vartheta$ admits a HCMM?
	
	\vfill
	\begin{tcolorbox}
		$\exists$ auxiliary complex encoding a HCMM
		\begin{displaymath}
			C_{\mathfrak{g}} := CE(\mathfrak{g})\otimes \Omega(M) = 
			\text{tot}\Big(\Lambda^{\geq 1} 
		\mathfrak{g}^*\otimes \Omega^\bullet(M),~ \delta_\text{CE},~d~\Big)
		\end{displaymath}
	\end{tcolorbox}

		HCMM are in 1:1 with primitives of a certain cocycle \hyperlink{frame:AppCohomoObstructions}{(constructed out of $\omega$ and $v$)}
	%
	\pause
	
\tcbset{colback=white,
	colbacktitle=white,
	colframe=blue!70!black,
	boxrule=1pt,
	colupper=blue!70!black,
	arc=15pt,
	}
\begin{tcolorbox}[sidebyside,righthand width=7cm]
	Prop: \cite{Ryvkin2015}+\cite{Fregier2015}
\tcblower
	$\exists$ HCMM 
	$\Leftrightarrow ~ \lbrack\tilde{\omega}\rbrack=0\in H^{n+1}(C_\mathfrak g)$
\end{tcolorbox}

	\pause
	\vfill
	Disadvantages:
	\begin{itemize}
		\item \danger purely algebraic \danger
		\item Desirable to translate it in a cohomology theory pertaining to the geometrical data.
	\end{itemize}


\end{frame}
\note[itemize]{
	\item idea: 	
		A HCMM is a sequence of multilinear maps $f_k \in Hom(\wedge^k \mathfrak{g},\Omega^{n-k}(M))\subset C_\mathfrak{g}^n$
	\item
		
	\item equivariant cohomology = "cohomology theory applying to topological spaces with group actions"

}
%-------------------------------------------------------------------------------------------------------------------------------------------------


%-------------------------------------------------------------------------------------------------------------------------------------------------
\begin{frame}[fragile]{Cohomological obstructions for compact groups}\label{frame:obstructioncompactgroups}
\tcbset{colback=white,
	colbacktitle=white,
	colframe=blue!70!black,
	boxrule=1pt,
	colupper=blue!70!black,
	arc=15pt,
	}
\begin{tcolorbox}[sidebyside,righthand width=7cm]
	Prop: \cite{Callies2016}+\cite{Fregier2015}
\tcblower
	Translation of $[\tilde{\omega}]$ in terms of the \emph{Bott-Shulman-Stasheff} and \emph{Cartan} models of 
	\tikz[baseline,remember picture]{\node[rounded corners,
                        fill=orange!10,draw=orange!30,anchor=base]            
            			(target) {\hyperlink{frame:EquivariantCohomology}{equivariant cohomology}};
            	}					
\end{tcolorbox}
	\pause
		\tikz[overlay,remember picture]
		{
			\node[rounded corners,
                 fill=orange!10,draw=orange!30,anchor=base]            
            	 (base) at ($(current page.north east)-(2.25,3.25)$) [rotate=-0,text width=4cm,align=center] {\footnotesize{cohomology theory for group actions on topological spaces}};
		}	
	\begin{tikzpicture}[overlay,remember picture]
    	\path[->] (base.north) edge[bend right,red](target.east);
    \end{tikzpicture}	

	\pause	
	\vfill
	Disadvantages:
	\begin{itemize}
		\item difficult to compute
		\item dependent on the models.
	\end{itemize}
	\pause

	\begin{tcolorbox}[
		enhanced,frame hidden,
		borderline={0.5pt}{0pt}{purple!70!black,dashed},
		sidebyside,righthand width=9cm,colframe=purple!70!black]	
	\alert{Goal:}
	\tcblower
	 Read obstructions in a simpler (more geometrical) cohomology theory (at least in simpler cases)
	\end{tcolorbox}
	\pause
	\vfill
	\vspace{1em}
	Let $\vartheta:G\times M\to M$ be a compact Lie group action:
	\begin{thmblock}[\cite{Miti2019}]
	\hyperlink{frame:cohomologicalproposition}{
	\vspace{-1em}
	$$\vartheta ~\text{admits HCMM}~ ~\Longleftrightarrow~  
	 ~[\vartheta^*\omega-\pi^*\omega]=0\in H^{n+1}_{dR}(G\times M)$$ 
	 }
	 \vspace{-1.5em}
	\end{thmblock}


\end{frame}
\note[itemize]{

	
	\item equivariant cohomology = "cohomology theory applying to topological spaces with group actions"

	\item corollaries:
		\begin{itemize}
			\item if $\omega$ has invariant potential $\Rightarrow$ $\exists$ HCMM.
			\item if $\omega$ can be extended to a \emph{equivariant cohomology class} $\Rightarrow$ $\exists$ HCMM.
		\end{itemize}
}
%-------------------------------------------------------------------------------------------------------------------------------------------------



%-------------------------------------------------------------------------------------------------------------------------------------------------
\begin{frame}{HCMM for actions on spheres}\label{frame:leoresults}
	Corollaries:
	\begin{itemize}
		\item if $\omega$ admits a $G$-invariant potential, then there is a HCMM;
		\item if $\tilde{\omega}$ can be extended to an equivariant cohomology class $\tilde{\omega}\in H_g^{k+1}(M)$ then there is a HCMM;
		\item If $G_i$ admits HCMM on $(M_i,\omega_i)$ for $i=1,2$ then $(M_1\times M_2, \pi_1^\ast \omega_1 + \pi_2^\ast \omega_2)$;
		% and $(M_1\times M_2, \pi_1^\ast \omega_1 \wedge \pi_2^\ast \omega_2)$ admit HCMM.
	\end{itemize}
	%
	\vfill
	\pause
	%
	\begin{thmblock}[\hyperlink{frame:LeoThmProof}{Classification of multisymplectic actions on spheres} \cite{Miti2019}]
		Let $M= (S^n,\omega)$ be the $n$-dimensional sphere with the standard volume,
		\\
		$\vartheta:G\times S^n \to S^n$ be an \emph{effective}, \emph{compact}, \emph{multisymplectic} action:
		\vspace{-.5em}
		\begin{displaymath}
			\vartheta ~\text{admits HCMM}~ ~\Longleftrightarrow~ 
			\begin{cases}
				n ~\text{is even}~
				\\
				\quad \text{\scriptsize{-\emph{ or }-}}
				\\
				\vartheta ~\text{is non-transitive}
			\end{cases}					
		\end{displaymath}
	\end{thmblock}	
	%
	\pause
	\begin{columns}[T]
		\begin{column}{.2\linewidth}
			Examples:
		\end{column}
		%
		\begin{column}{.6\linewidth}
			\begin{itemize}
				\item[\cmark] \hyperlink{frame:TransExample}{$SO(2n+1) \circlearrowleft S^{2n}$}
				\item[\xmark] $SO(2n) \circlearrowleft S^{2n-1}$
				\item[\cmark] \hyperlink{frame:LeoNonTransExample}{$SO(n) \circlearrowleft S^{n}$}
			\end{itemize}				
		\end{column}
	\end{columns}


\end{frame}
\note[itemize]{
	\item	
	Unlike the symplectic case, the converse statement does not hold in general. 
	Even if a (pre-)multisymplectic action of $G$ on $(M,\omega)$ admits a comoment, $[\omega]$ does not need to come from an equivariant cocycle. 
	
	\item Consider: $S^n$ multisymplectic w.r.t the standard volume 
		$\omega$.
	\item $G$ compact Lie group acting effectively and preserving the volume.
		This action admits HCMM if and only if $n$ is even or the action is not transitive.	
	\item The obstructions found do not prevent existence of \emph{weak homotopy co moment maps}.
}
%-------------------------------------------------------------------------------------------------------------------------------------------------






%----------------------------------------------------------------------------------------------------------------------------------
\end{document}
%----------------------------------------------------------------------------------------------------------------------------------




